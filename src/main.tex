%! Author = mddzi
%! Date = 02.06.2024

% Preamble
% czcionka zgodna z wytycznymi to 10pt
\documentclass[12pt,a4paper,twoside]{report}

% Packages
\usepackage{amsmath}
\usepackage[T1]{fontenc}
\usepackage[utf8]{inputenc}
\usepackage{lmodern}
\usepackage{textcomp}
\usepackage{lastpage}
\usepackage[inner=3.5cm,outer=2.5cm,top=2.5cm,bottom=2.5cm]{geometry} % margines zgodny z wytycznymi
%\usepackage[margin=3.5cm]{geometry} % margines oryginalny
\usepackage{graphicx}
\usepackage{fancyhdr}
\usepackage[svgnames]{xcolor}
\usepackage[font={small,color=Grey},labelsep=period,width={0.8\textwidth}]{caption}
\usepackage{float}
\usepackage{multirow}
\usepackage{subcaption}
\usepackage{adjustbox}
\usepackage{longtable}
\usepackage{etoolbox}
\usepackage{hyperref}
\usepackage{pdfpages}
\usepackage{indentfirst}
\usepackage{icomma}
\usepackage{natbib}
\usepackage{microtype}
\usepackage{url}
\usepackage{calc}
\usepackage{enumitem}
%\usepackage{lstmisc}
%\usepackage[backend=bibtex, sorting=none]{biblatex}

% set url breaks
\def\UrlBreaks{\do\/\do-}

% figures count
\counterwithout{figure}{chapter}

% specified names
\renewcommand{\figurename}{Rysunek}
\renewcommand{\chaptername}{Rozdział}
\renewcommand{\contentsname}{Spis treści}
\renewcommand{\bibname}{Bibliografia}
\renewcommand{\listfigurename}{Spis rysunków}
\renewcommand{\baselinestretch}{1.1} % interlinia oryginalna
%\renewcommand{\baselinestretch}{1.5} % interlinia zgodna z wytycznymi
\setlength{\parindent}{1.25cm} % wcięcie akapitowe zgodne z wytycznymi

% custom compact enumarate
\newenvironment{cenumerate}
{ \begin{enumerate}[itemsep=5pt,topsep=5pt,parsep=0pt] }
{
    \end{enumerate}
    \vspace{5pt}
}

% custom compact itemize
\newenvironment{citemize}
{ \begin{itemize}[itemsep=5pt,topsep=5pt,parsep=0pt] }
{
    \end{itemize}
    \vspace{5pt}
}

%define the environment for compact enumeration
%\newenvironment{cenumerate}{
%\setlength{\parskip}{0pt}
%\begin{enumerate}
%\setlength{\itemsep}{0pt}
%\setlength{\parskip}{0pt}
%\setlength{\parsep}{0pt}
%}
%{\end{enumerate}\setlength{\parskip}{11pt}}
%%define the environment for compact itemization
%\newenvironment{citemize}{
%\setlength{\parskip}{0pt}
%\begin{itemize}
%\setlength{\itemsep}{0pt}
%\setlength{\parskip}{0pt}
%\setlength{\parsep}{0pt}
%}
%{\end{itemize}\setlength{\parskip}{11pt}}


% set pagestyle
\pagestyle{fancy}

\title{\textbf{Program do tworzenia schematów układów scalonych w formie gry edukacyjnej}\\[2ex]
    \large Program for creating IC diagrams in the form of an educational game\\
}
\author{Maksymilian Dziemiańczuk}
\date{}

% Header and footer
\fancyhf{}
\fancyhead{}
\fancyhead[RO,LE]{Maksymilian Dziemiańczuk}
\fancyhead[RO,LE]{Projekt inżynierski}
\fancyfoot{}
\fancyfoot[RO,LE]{\thepage}
\fancyfoot[RE,LO]{\footnotesize Program do tworzenia schematów układów scalonych w formie gry edukacyjnej}
\fancypagestyle{plain}{
    \renewcommand{\headrulewidth}{0pt}
    \fancyhf{}
    \fancyfoot[RO,LE]{\thepage}
    \fancyfoot[RE,LO]{\footnotesize Program do tworzenia schematów układów scalonych w formie gry edukacyjnej}
}

% Document
\begin{document}

% title page
\includepdf[pages={1}]{StronaTytulowa_185578.pdf}

% empty page
\newpage \thispagestyle{empty} \ \newpage

\section*{Streszczenie}
\label{sec:streszczenie}
\thispagestyle{plain}
%\abstract
Celem niniejszej pracy jest opracowanie edukacyjnej gry komputerowej umożliwiającej wprowadzenie użytkownika
w zagadnienia projektowania schematów układów scalonych,
z wykorzystaniem silnika Unity i języka C\#.
Aplikacja stanowi narzędzie zarówno do nauki,
jak i praktycznego ćwiczenia podstawowych koncepcji projektowania w interaktywnej formie,
łącząc aspekty edukacyjne z rozrywką.\\
\indent %Do realizacji projektu przyjęto architekturę MVVM (Model-View-ViewModel),
%która zapewnia modularność oraz przejrzystość kodu,
%umożliwiając łatwe utrzymanie i przyszłe rozszerzenia aplikacji.
Kluczowym elementem było opracowanie algorytmu weryfikującego
poprawność tworzonych schematów oraz system poziomów o rosnącej trudności.
%opartego na cyklu: spełnienie wymagań poziomu – projektowanie układu – testowanie poprawności – ocenianie – \linebreak
%- przejście
%do kolejnego poziomu.
%Poszczególne poziomy stanowią narastające wyzwania,
%które pozwalają użytkownikowi rozwijać swoje umiejętności projektowe.\\
Innym istotnym aspektem pracy było zaprojektowanie intuicyjnego edytora schema\-tów,
dostosowanego do potrzeb początkujących użytkowników.
%W edytorze uwzględniono mechanizmy umożliwiające dynamiczne zarządzanie widocznością poszczególnych warstw i połączeń,
%co ułatwia zrozumienie oraz kontrolę nad tworzonymi projektami.\\
W edytorze zaimplementowano obsługę skrótów klawiszowych, oraz zaprojektowano intuicyjne w obsłudze narzę\-dzia rysowania i edycji,
zapewniające płynne i szybkie tworzenie schematów.\\
\indent Efektem jest funkcjonalna gra edukacyjna,
która oferuje użytkownikom narzę\-dzie do nauki projektowania układów scalonych
i umożliwia stopniowe doskonalenie umiejętności wraz z podnoszeniem poziomu trudności.\\
\\


\section*{Abstract}
\label{sec:abstract}

The aim of this thesis is to develop an educational computer game
that introduces users to the principles of designing integrated circuit schematics,
utilizing the Unity engine and the C\# programming language.
The application serves as a tool for both learning and practicing basic design concepts in an interactive format,
effectively combining educational and entertainment aspects.\\
\indent A key feature of the project is the development of an algorithm to verify the correctness of created schematics
and a level system with progressively increasing difficulty.
Another important aspect of the work is the design of an intuitive schematic editor tailored to the needs of novice users.
The editor integrates support for keyboard shortcuts and intuitive drawing and editing tools,
ensuring smooth and efficient schematic creation.\\
\indent The result is a functional educational game that provides users with a comprehensive tool
for learning integrated circuit design while
facilitating gradual skill improvement through progressively challenging levels.


\tableofcontents

\chapter{Wstęp i cel pracy}

Programy do tworzenia schematów są jednym z~kluczowych elementów
procesu wielkoskalowej integracji układów scalonych (ang. VLSI — Very Large Scale Integration)~\cite{VLSI_insemi}.
Polega to na projektowaniu podukładów o określonych funkcjach,
które następnie są łączone w~jeden, w~pełni działający układ scalony.
Taki proces upraszcza projektowanie oraz produkcję układów scalonych, ponieważ prace można podzielić na mniejsze,
mniej skomplikowane części.
Dodatkowo wiele podukładów może być wykorzystywanych w~różnych projektach,
co~znacznie zwiększa efektywność pracy i~obniża jej koszty.
W~efekcie,
programy wspierające projektowanie schematów są nieodzownym narzędziem przy tworzeniu układów scalonych.
%w~ten sposób upraszczany jest proces projektowania oraz produkcji układów scalonych,
%gdyż pracę można podzielić na mniejsze części,
%a~dodatkowo cześć podukładów może być używana w~wielu różnych projektach.
Przykładem takiego programu jest open-source'owy (o publicznym kodzie źródłowym),
Magic VLSI stworzony przez Johna Ousterhouta w~1980 roku,
napisany w~języku C na platformę Linux~\cite{MAGIC_article}.
Charakteryzuje go prosta szata graficzna oraz szeroki zakres działania,
lecz jego obsługa bywa często nieintuicyjna oraz nieprzyjazna dla początkujących użytkowników.
Kolejnym powszechnie dostępnym programem jest Microwind,
na który składa się zestaw modułów,
jednym z~nich jest edytor schematów~\cite{Microwind}.
Niestety, aby używać pełnej i~aktualnej wersji oprogramowania potrzebna jest licencja.
W~wielu innych przypadkach programy te~są~dostępne jedynie w~wersjach płatnych, kierowanych do dużych firm.
To właśnie wysoki próg wejścia
i~złożoność obsługi tego rodzaju programów stały się inspiracją do opracowania tego projektu.
%\indent Celem tej pracy było stworzenie
%programu edukacyjnego w~formie gry, który pomoże w~nauce tworzenia schematów układów scalonych,
%ponieważ już same gry komputerowe, nawet te~niemające wyraźnie edukacyjnego charakteru,
%mogą rozwijać umiejętności graczy dzięki stawianiu przed nimi angażujących wyzwań
%i~zachęcaniu do rozwiązywania problemów w kreatywny sposób~\cite{videogames}.\\
\vspace*{0.1cm}

\indent Celem tej pracy jest opracowanie programu edukacyjnego w~formie gry z~wykorzystaniem silnika Unity,
który pomoże w nauce tworzenia schematów układów scalonych,
poprzez prosty i~intuicyjny edytor oraz system rozgrywki.

\vspace*{0.1cm}

\indent Jedne z~głównych założeń programu to prostota obsługi, intuicyjność oraz ergonomia,
na które mają wpływ przede wszystkim dobrze zaprojektowany interfejs graficzny użytkownika
oraz wbudowane narzędzia wspierające edycję schematu.
%Na tych założeniach opiera się cała część projektu związana z~interfejsem użytkownika.\\
Funkcje te~stanowią podstawę części projektu związaną z~interfejsem użytkownika
i~odpowiadają za poprawę użyteczności
oraz dostępność aplikacji.\\
\indent Gry komputerowe, nawet te~niemające wyraźnie edukacyjnego charakteru,
mogą rozwijać umiejętności graczy.
Osiągają to~poprzez stawianie przed nimi angażujących wyzwań
oraz zachęcanie do kreatywnego rozwiązywania problemów~\cite{videogames}.
Stąd kolejnym istotnym elementem tej pracy jest przygotowanie angażującej
i~skalowalnej pętli rozgrywki,
która pozwoli na stopniowe wprowadzanie użytkownika w~proces projektowania schematów układów scalonych,
wraz z~implementacją systemów sprawdzania poprawności wykonanych zadań, wskazywania błędów i~ewentualnych podpowiedzi.
%Dzięki realizacji założeń program pomoże w~nauce podstaw bez odrzucania użytkownika
%przez zbyt skomplikowaną obsługę.\\
Dzięki realizacji założeń, projekt ten pozwoli użytkownikowi na zdobycie podstawowej wiedzy
i~umiejętności z~zakresu projektowania układów scalonych,
bez konieczności przechodzenia przez trudniejsze w~obsłudze, tradycyjne oprogramowanie.\\
\indent Zważając na wszystkie założenia oraz charakter edukacyjno-growy projektu,
do realizacji wybrano silnik Unity Engine,
bazujący na języku C\#,
powszechnie wykorzystywany w~tworzenia gier komputerowych.
%program został stworzony w~silniku Unity Engine,
%powszechnie używanym do tworzenia gier komputerowych.
%natomiast możliwość łatwego tworzenia interaktywnej aplikacji graficznej sprawdzi się w~tego typu zastosowaniach.\\
Wybór Unity podyktowany był prostotą tworzenia aplikacji graficznych oraz łatwego tworzenia interaktywnych elementów,
co~można wykorzystać nie tylko w~grach, ale również w~różnego typu narzędziach.
Dzięki Unity,
program łączy funkcje gry z~elementami edukacyjnymi,
tworząc przystępną i~atrakcyjną platformę do nauki podstaw projektowania układów scalonych.
%Został on wybrany ze względu na prostotę tworzenie aplikacji graficznych oraz łatwego tworzenia interaktywnych elementów,
%coi~można wykorzystać nie tylko w~grach, ale również w~różnego typu narzędziach.
%Aby ujednolicić schematy oraz zadania do wykonania dla użytkowników,
%projektowanie będzie odbywać się w~tylko jednej technologii\ \textendash \ CMOS AMIS ami-C5.
%Powodem wyboru tej technologii jest wcześniejsze doświadczenie w~projektowaniu schematów układów scalonych
%w~tej technologii.

\chapter{Przegląd istniejących rozwiązań}
\label{ch:przeglad_istniejacych_rozwiazan}

Projektowanie topografii jest obecnie istotnym elementem procesu tworzenia współczesnej elektroniki,
wymagającym zarówno solidnej wiedzy teoretycznej,
jak~i~umiejętności praktycznych korzystania z~odpowiednich narzędzi.
Dostępne na rynku programy wspierające ten proces oferują wiele zaawansowanych funkcji,
ale ich złożoność oraz częsta potrzeba posiadania płatnej licencji ogranicza ich przystępność dla osób początkujących.\\
\indent Istniejące rozwiązania mogą wskazać jakie funkcje, zarówno podstawowe, jak~i~zaawansowane,
powinny zostać zaimplementowane w~programie.
Analiza ich zalet, ograniczeń, jak również przystępności dla użytkowników pozwala określić,
w~jaki sposób unikać problemów z~intuicyjnością i~ergonomią interfejsu jednocześnie wykorzystując sprawdzone mechanizmy.\\
\indent Poza programami do tworzenia schematów należy również przyjrzeć się innym profesjonalnym lub półprofesjonalnym narzędziom,
co pozwoli na zrozumienie,
jak powinien być zaprojektowany schludny i~intuicyjny interfejs użytkownika.

\section{Magic VLSI}

Jednym z bardziej popularnych oraz najbardziej dostępnych programów do projektowania układów scalonych
jest Magic VLSI\@.
Opracowany w 1983 roku przez Johna K. Ousterhouta i jego zespół na Uniwersytecie Kalifornijskim w Berkeley,
napisany w języku C, pierwotnie dla systemu Berkeley 4.2 będącego wariacją platformy Unix~\cite{MAGIC_article}.
Program ten posiada własną stronę internetową, na której można znaleźć dokumentację, kod źródłowy,
wskazówki dotyczące projektowania schematów, jak również pobrać program~\cite{MAGIC_site}.
Dzięki otwartemu kodowi źródłowemu, z czym wiąże się brak konieczności posiadania płatnej licencji,
oraz wysokiej funkcjonalności pozwalającej na pełne zaprojektowanie schematu układu scalonego,
Magic zyskał dużą popularność
w środowiskach akademickich i naukowych, a także wśród hobbystów.
Należy jednak zaznaczyć, że szeroka funkcjonalność programu
niekoniecznie sprzyja pozytywnym wrażeniom pierwszego użytkowania \textit{FTUE}
(ang. \textit{First Time User Experience}).
\newpage
\indent Unikalną cechą programu Magic jest wprowadzenie techniki strukturyzacji danych
zszytych narożników \textit{corner-stitched},
która znacząco poprawia jego wydajność.
Mechanizm ten opiera się na reprezentacji układu scalonego jako zestaw warstw,
na które składa się zestaw prostokątnych komórek.
%(ang. \textit{cells}).
Każda komórka zawiera zszyte narożnikami powierzchnie (ang. \textit{corner-stitched planes}),
opisujące jej geometrię oraz podkomórki, a każda z nich składa się z wielu prostokątnych kafelków.
Taka struktura pozwala na szybkie i efektywne operacje na schemacie,
takie jak znajdywanie wszystkich kafelek w danym obszarze lub sąsiadów danej komórki.
Dzięki zastosowanym mechanizmom możliwe jest także wykonywanie operacji na dużych obszarach
i szybką aktualizację danych.\\
\indent Magic posiada prosty interfejs graficzny,
który składa się z obszaru roboczego, paska menu oraz paska wyboru materiałów,
przedstawione na rys.~\ref{fig:magic_okno}.

\begin{figure}[h]
    \centering
    \includegraphics[width=.9\textwidth]{chapters/chapter2/img/magic_okno}
    \caption[Widok głównego okna programu Magic VLSI.]{Widok głównego okna programu Magic VLSI, źródło:~\cite{MAGIC_site}.}
    \label{fig:magic_okno}
\end{figure}

\indent Dodatkowo wraz z głównym oknem programu otwiera się okno konsoli pozwalające na wprowadzanie dodatkowych komend,
przedstawione na rys.~\ref{fig:magic_konsola}.

\begin{figure}[h]
    \centering
    \includegraphics[width=.9\textwidth]{chapters/chapter2/img/magic_okno_konsola}
    \caption[Widok okna konsoli programu Magic VLSI.]{Widok okna konsoli programu Magic VLSI, źródło:~\cite{MAGIC_site}.}
    \label{fig:magic_konsola}
\end{figure}

\indent Rysowanie zaczyna się od zaznaczenia obszaru, lewy przycisk myszy służy do wyboru pozycji obszaru,
a prawy do jego rozszerzania.
Aby wypełnić obszar, należy wybrać odpowiedni materiał z paska wyboru
lub z już narysowanych komórek,
poprzez kliknięcie środkowym przyciskiem myszy.
Jest to początkowo nieintuicyjne dla użytkownika,
natomiast charakteryzuje się precyzją, stąd podobna funkcjonalność została zaimplementowana w projekcie.
W przypadku reszty funkcji takich jak przesuwanie, kopiowanie i skalowanie komórek,
jak również sama nawigacja po obszarze roboczym odbiega od obecnego standardu różnego rodzaju edytorów
i nierzadko wymaga wpisywania komend w konsoli.
Jest to obecnie rozwiązanie nieergonomiczne, gdyż obecnie większość podstawowych operacji wykonywanych jest
jedynie przy użyciu skrótów klawiszowych oraz myszy.\\
\indent Kolejnym istotnym aspektem jest także sama instalacja programu Magic VLSI\@.
Ze względu na pierwotne przeznaczenie programu dla systemu Unix,
jego instalacja na systemach Windows wymaga albo zainstalowania maszyny wirtualnej z systemem Linux,
albo skorzystania z dodatkowego narzędzia \textit{Cygwin}~\cite{MAGIC_site,cygwin}.
Zważając na dużo większą popularność systemów operacyjnych Windows~\cite{os_stats},
może być to niepotrzebną komplikacją dla użytkowników dopiero uczących się projektowania układów scalonych.\\
\indent Nadmienić jednak trzeb, że istotną zaletą programu Magic jest prosta szata graficzna,
a w szczególności tekstury materiałów, które są czytelne, pozwalają na łatwe rozróżnienie warstw,
nawet gdy się na siebie nakładają, dzięki dobrze dobranym wzorom i kolorom, przykład przedstawiono na rys. \ref{fig:magic_tran}.
Implementacja podobnego rozwiązania w projekcie poprawi czytelność schematów.

\begin{figure}[h]
    \centering
    \includegraphics[width=.9\textwidth]{chapters/chapter2/img/magic_tran}
    \caption[Przykład tranzystora pMOS narysowanego w programie Magic VLSI\@.]
    {
        Przykład tranzystora pMOS narysowanego w programie Magic VLSI\@.
        Mimo nakładania się wielu warstw schemat jest dalej czytelny,
        nawet w miejscu połączenia warstw~\textit{metal1} i~\textit{pdiffusion} oraz warstwy~\textit{nwell},
        żródło: opracowanie własne.
    }
    \label{fig:magic_tran}
\end{figure}

\section{Microwind}

Microwind to zintegrowane oprogramowanie należące do rodziny EDA (ang. \textit{Electronic Design Automation}),
służące do automatyzacji procesu projektowania układów scalonych lub płytek drukowanych,
umożliwiające projektowanie, symulacje, weryfikacje oraz testowanie układów elektronicznych~\cite{eda}.
Program ten został opracowany przez dra Sicarda do celów edukacyjnych, składa się z kilku modułów,
odpowiadających za różne etapy projektowania układów scalonych~\cite{Microwind}.
Instalacja w przypadku Microwinda jest prosta, wymaga jedynie pobrania pliku instalacyjnego z oficjalnej strony,
a następnie zainstalowania go na komputerze z systemem Windows~\cite{Microwind},
są to natomiast wersje lite (wersja z ograniczoną funkcjonalnością), pełna wersja wymaga licencji.
Dostępna jest także archiwalna pełna wersja programu, która jest dostępna za darmo~\cite{old_microwind}.
Strona zawiera również dokumentację oraz przykłady projektów,
które można wykorzystać w celach edukacyjnych~\cite{Microwind}.\\
\indent Jednym z modułów programu Microwind jest edytor schematów \textbf{Nano Lambda},
pojawiający się domyślnie po uruchomieniu programu.
Posiada dość nieskomplikowany interfejs graficzny z paskiem menu i narzędzi
oraz pływającym oknem palety warstw, przedstawiony na rys.~\ref{fig:microwind_okno}.

\begin{figure}[h]
    \centering
    \includegraphics[width=.9\textwidth]{chapters/chapter2/img/microwind_okno}
    \caption[Widok głównego okna programu Microwind.]{Widok głównego okna programu Microwind, źródło:~\cite{Microwind}.}
    \label{fig:microwind_okno}
\end{figure}

\indent Rysowanie odbywa się podobnie jak w przypadku klasycznych edytorów graficznych,
poprzez wybór warstwy z palety,
a następnie przeciągając kursorem po obszarze roboczym przy wciśniętym lewym lub środkowym przycisku myszy,
rysując przy tym prostokątną komórkę.
Nieznaczną wadą rysowania w Microwindzie jest brak przyciągania do siatki,
przez co jest mało precyzyjne.
Przemieszczanie się na obszarze roboczym wymaga używania klawiszy kierunkowych lub przycisków na pasku narzędzi,
co obecnie jest już rozwiązaniem nieergonomicznym.
Program Microwind, poza typowymi narzędziami edycji, charakteryzuje się wieloma zautomatyzowanymi narzędziami,
pozwalającymi generowanie elementów schematów,
na przykład na podstawie funkcji logicznych lub kodu Verilog~\cite{microwind_operation_commands}.\\
\indent Ze względu na brak otwartego kodu źródłowego trudno określić dokładnie zastosowane mechanizmy edytora,
natomiast na podstawie obserwacji można stwierdzić,
że każda edycja schematu wywołuje ponowne rysowanie całego obszaru roboczego,
co jest szczególnie zauważalne podczas usuwania komórek.
Ta sama operacja również wskazuje na strukturę danych, która zakotwiczeniu komórek w kolumnach,
ponieważ po usunięciu obszaru wewnątrz większej komórki, cała kolumna zostaje usunięta.\\
% TODO: poniższe poprawić
\indent Szata graficzna samego edytora charakteryzuje się wysokim kontrastem,
gdzie warstwy są jednolitymi kolorami, częściowo przezroczystymi,
co jest zauważalne, gdy warstwy te się nakładają, przykład przedstawiono na rys.~\ref{fig:microwind_tran}.

\begin{figure}[h]
    \centering
    \includegraphics[width=.9\textwidth]{chapters/chapter2/img/microwind_tran}
    \caption[Przykład tranzystora narysowanego w programie Microwind.]
    {
        Przykład tranzystora narysowanego w programie Microwind,
        warstwy \textit{Metal 1}, \textit{P+ Diffusion} oraz \textit{Polysilicon} są jednoilitymi kolorami,
        z wyjątkiem warstwy \textit{N Well} którą reprezentuje kropkowany wzór,
        , źródło: opracowanie własne.
    }
    \label{fig:microwind_tran}
\end{figure}

\section{Electric}

Kolejnym programem do projektowania schematów układów scalonych jest Electric,
opracowany przez S. M. Rubina w 1981 roku.
Od samego początku posiadał otwarty kod źródłowy
oraz był udostępniany wielu uczelniom na całym świecie.
Oryginalnie napisany w języku C,
natomiast w 2003 rozpoczęto proces tłumaczenia na język Java, zakończony sukcesem w 2005 roku~\cite{electric_gnu}.
Dzięki temu program jest dostępny na każdy system,
na którym zainstalowana jest Java w wersji 1.6 lub więcej.
Do samej instalacji wystarczy natomiast pobrać plik \texttt{.jar} z oficjalnej strony programu~\cite{electric_sfs}.
Podobnie jak Microwind, Electric jest oprogramowaniem EDA, zawierający wiele modułów wspierających projektowanie układów scalonych.
Większość nich jest pod postacią dodatkowo instalowanych wtyczek,
poszerzających jego funkcjonalność~\cite{electric_sfs, electric_gnu}.\\
\indent Electric charakteryzuje się całkowicie odmienną metodą projektowania topografii niż większość programów.
Podobnie do edytorów schematów elektrycznych, używa podejścia połączeniowego,
w przeciwieństwie do reszty, gdzie występuje podejście geometryczne, czyli rysowanie prostokątów.
Oparcie projektowania o metodę połączeń polega na wstawianiu konkretnych elementów będących węzłami,
a następnie łączenie ich ze sobą.
Na przykład tranzystor jest już gotowym elementem, który można wstawić na schemat,
którego nie trzeba specjalnie rysować, jak to jest w przypadku programów Magic czy Microwind.
Upraszcza to cały proces projektowania,
natomiast problem może stanowić tworzenie układów o bardziej skomplikowanej i złożonej geometrii.\\
\indent Dzięki wykorzystaniu Javy, Electric posiada nowocześniejszy interfejs graficzny,
pokazany na rys.~\ref{fig:electric_okno}.
Także sposób poruszania się po obszarze oraz skróty klawiszowe są bardziej zgodne z obecnymi standardami,
dzięki czemu program jest bardziej intuicyjny w obsłudze.

\begin{figure}[h]
    \centering
    \includegraphics[width=.9\textwidth]{chapters/chapter2/img/electric_okno}
    \caption[Widok głównego okna programu Electric.]{Widok głównego okna programu Electric, źródło:~\cite{electric_sfs}.}
    \label{fig:electric_okno}
\end{figure}

Prócz stałych pasków narzędzi i menu, program opiera się na pływających oknach,
dzięki czemu można jednocześnie mieć wyświetlone kilka narzędzi jednocześnie.
Na rys.~\ref{fig:electric_okno} przedstawiono okno do edycji schematu oraz okno dziennika zdarzeń.
Edytor zamiast palety warstw posiada okno z dostępnymi elementami, połączeniami i kontaktami pomiędzy warstwami.
W przypadku Electrica, do przedstawiania warstw wykorzystuje się jednolite kolory, oraz wzory,
co przedstawiono na rys.~\ref{fig:electric_tran}.
Przy edycji oraz rysowaniu topografii można zauważyć miganie elementów,
co wskazuje, że cały widok jest na nowo renderowany po każdej edycji lub przesunięciu.

\begin{figure}[h]
    \centering
    \includegraphics[width=.9\textwidth]{chapters/chapter2/img/electric_tran}
    \caption[Przykład tranzystora pMOS narysowanego w programie Electric]
    {
        Przykład tranzystora pMOS narysowanego w programie Electric,
        źródło: opracowanie własne.
    }
    \label{fig:electric_tran}
\end{figure}

\section{Inne programy}
\label{sec:inne_programy}

Podczas opracowywania oprogramowania,
które spełniać będzie obecne standardy ergonomii i~intuicyjności,
warto zwrócić uwagę na inne popularne programy z~odrębnych kategorii.
W~szczególności będącym na darmowej licencji,
przez~co~są dostępne dla większej liczby użytkowników
i~idealne do nauki danej dziedziny, której dotyczą.

\subsection{GIMP}
\label{subsec:gimp}

Podobnie jak Electric, GIMP jest oficjalnym pakietem GNU i~również posiada otwarty kod źródłowy
oraz~stosuje założenia wolnego oprogramowania~\cite{gimp_site}.
Jest dostępny na wiele platform, w~tym na systemy Windows, Linux oraz~macOS.
GIMP to program do edycji grafiki rastrowej.
Charakteryzuje się przejrzystym interfejsem graficznym oraz~dużą ilością narzędzi,
gdzie każde z~nich ma swój własny skrót klawiszowy, co~znacznie ułatwia pracę w~programie.
\indent Interfejs graficzny jest podzielony na trzy główne panele, nie licząc obszaru roboczego:

\begin{citemize}
    \item Górny panel menu programu -- dostępne opcje dla programu oraz~dodatkowe opcje do modyfikacji edytowanego obrazu,
    \item Prawy panel -- zawiera przede wszystkim panel do zarządzania warstwami obrazu,
    ale również szereg innych opcji takich jak krzywe, desenie czy wybór pędzli
    \item Lewy panel narzędzi -- zawiera narzędzia do rysowania oraz modyfikacji obrazu,
    a~także opcje tych narzędzi.
\end{citemize}

Widok okna programu GIMP przedstawiono na rys.~\ref{fig:gimp_okno}.
Dzięki zastosowaniu takiego interfejsu program jest przejrzysty dzięki zgrupowaniu konkretnych typów funkcjonalności,
co~może mieć także zastosowanie w~opracowywanym programie. \\
\indent Dzięki darmowemu dostępowi i~szerokiej funkcjonalności,
GIMP został także wykorzystany do opracowania grafik do niniejszej pracy.

\begin{figure}[h]
    \centering
    \includegraphics[width=.9\textwidth]{chapters/chapter2/img/gimp}
    \caption[Widok głównego okna programu GIMP.]{Widok głównego okna programu GIMP, źródło:~\cite{gimp_site}.}
    \label{fig:gimp_okno}
\end{figure}

\subsection{Blender}
\label{subsec:blender}

Program Blender jest kolejnym przykładem oprogramowania z~otwartym kodem źródłowym na licencji GNU~\cite{blender_site}.
Wykorzystywany jest do tworzenia grafiki trójwymiarowej, animacji oraz~efektów specjalnych.
Ze względu na rozbudowane możliwości oraz~ilość narzędzi i~funkcji,
zasadnicze znaczenie ma ergonomia i~przejrzystość interfejsu, pokazanego na rys.~\ref{fig:blender_okno}.
Poza podobnymi jak w~reszcie programów panelami górnym i~prawym,
Blender wyróżnia się dolnym panelem z~osią czasu do animacji, a~także małym paskiem narzędzi w~lewym górnym rogu okna,
co~jest efektywnym sposobem pozwalającym na zaoszczędzenie miejsca na obszar roboczy,
gdy nie ma dużej liczby narzędzi do wykorzystania w~danym momencie.
Podobne rozwiązanie może znaleźć zastosowanie w~edytorze schematów.

\begin{figure}[h]
    \centering
    \includegraphics[width=.9\textwidth]{chapters/chapter2/img/blender}
    \caption[Widok głównego okna programu Blender.]{Widok głównego okna programu Blender, źródło:~\cite{blender_site}.}
    \label{fig:blender_okno}
\end{figure}



% Bibliography
\bibliography{main}
\bibliographystyle{unsrt}
\addcontentsline{toc}{chapter}{Bibliografia}
    
% List of figures
\thispagestyle{empty}
\listoffigures
\addcontentsline{toc}{chapter}{Spis rysunków}

\end{document}
