%! Author = mddzi
%! Date = 02.06.2024

% Preamble
\documentclass[12pt,a4paper,twoside]{report}

% Packages
\usepackage{amsmath}
\usepackage[T1]{fontenc}
\usepackage[utf8]{inputenc}
\usepackage{lmodern}
\usepackage{textcomp}
\usepackage{lastpage}
\usepackage{geometry}
\usepackage{graphicx}
\usepackage{fancyhdr}
\usepackage[svgnames]{xcolor}
\usepackage[font={small,color=Grey},labelsep=period,width={0.8\textwidth}]{caption}
\usepackage{float}
\usepackage{multirow}
\usepackage{subcaption}
\usepackage{adjustbox}
\usepackage{longtable}
\usepackage{etoolbox}
\usepackage{hyperref}
\usepackage{pdfpages}

\geometry{margin=3.5cm}

\renewcommand{\figurename}{Rysunek}
\renewcommand{\chaptername}{Rozdział}
\renewcommand{\contentsname}{Spis treści}
\renewcommand{\bibname}{Bibliografia}

\pagestyle{fancy}

\title{\textbf{Program do tworzenia schematów układów scalonych w formie gry edukacyjnej}\\[2ex]
    \large Program for creating IC diagrams in the form of an educational game\\
}
\author{Maksymilian Dziemiańczuk}
\date{}

\fancyhf{}
\fancyhead{}
\fancyhead[RO,LE]{Maksymilian Dziemiańczuk}
\fancyhead[RO,LE]{Projekt inżynierski}
\fancyfoot{}
\fancyfoot[RO,LE]{\thepage}
\fancyfoot[RE,LO]{\footnotesize Program do tworzenia schematów układów scalonych w formie gry edukacyjnej}
\fancypagestyle{plain}{
    \renewcommand{\headrulewidth}{0pt}
    \fancyhf{}
    \fancyfoot[RO,LE]{\thepage}
    \fancyfoot[RE,LO]{\footnotesize Program do tworzenia schematów układów scalonych w formie gry edukacyjnej}
}

% Document
\begin{document}

%\maketitle
\includepdf[pages={1}]{StronaTytulowa_185578.pdf}

%pusta strona
\newpage \thispagestyle{empty} \ \newpage

\section*{Streszczenie}
\thispagestyle{plain}

Lorem ipsum dolor sit amet, consectetur adipiscing elit

\tableofcontents

\chapter{Wstęp i cel pracy}
\section{Abstrakt}

Programy do tworzenia schematów są jednym z kluczowych elementów
procesu wielko-skalowej integracji układów scalonych (ang. VLSI — Very Large Scale Integration)\cite{VLSI}.
Polega to na projektowaniu podukładów o określonych funkcjach,
które następnie są łączone w jeden w pełni działający układ scalony.
W ten sposób upraszczany jest proces projektowania oraz produkcji układów scalonych,
ponieważ cześć podukładów może być używana w wielu różnych projektach
oraz łatwiej jest rozłożyć pracę na mniejsze części.
Przykładem takiego programu jest open-sourcowy, Magic VLSI stworzony przez Johna Ousterhouta w 1980 roku\cite{MAGIC},
napisany w języku C na platformę Linux\cite{MAGIC_wiki}.
Charakteryzuje go prosta szata graficzna oraz szeroki zakres działania,
natomiast jego obsługa bywa często nieintuicyjna oraz nieprzyjazna dla początkujących użytkowników
co stało się inspiracją dla tego projektu.

\section{Cel pracy}

%\LaTeX{} \cite{VLSI} is a set of macros built atop \TeX{} \cite{texbook}.

    \bibliography{main}
    \bibliographystyle{plain}

\end{document}
