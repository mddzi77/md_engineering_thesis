Programy do tworzenia schematów są jednym z kluczowych elementów
procesu wielko-skalowej integracji układów scalonych (ang. VLSI — Very Large Scale Integration)\cite{VLSI}.
Polega to na projektowaniu podukładów o określonych funkcjach,
które następnie są łączone w jeden w pełni działający układ scalony.
W ten sposób upraszczany jest proces projektowania oraz produkcji układów scalonych,
ponieważ cześć podukładów może być używana w wielu różnych projektach
oraz łatwiej jest rozłożyć pracę na mniejsze części.
Przykładem takiego programu jest open-sourcowy, MAGIC VLSI stworzony przez Johna Ousterhouta w 1980 roku\cite{MAGIC},
napisany w języku C na platformę Linux\cite{MAGIC_wiki}.
Charakteryzuje go prosta szata graficzna oraz szeroki zakres działania,
natomiast jego obsługa bywa często nieintuicyjna oraz nieprzyjazna dla początkujących użytkowników
co stało się inspiracją dla tego projektu.
Ze względu na wysoki próg wejścia programów takich jak MAGIC, celem tego projektu jest stworzenie
programu, który pomoże w nauce tworzenia schematów układów scalonych poprzez grę.