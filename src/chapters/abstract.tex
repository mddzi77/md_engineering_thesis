\section*{Streszczenie}
\label{sec:streszczenie}
\thispagestyle{plain}
%\abstract
Celem niniejszej pracy jest opracowanie edukacyjnej gry komputerowej umożliwiającej wprowadzenie użytkownika
w zagadnienia projektowania schematów układów scalonych,
z wykorzystaniem silnika Unity i języka C\#.
Aplikacja stanowi narzędzie zarówno do nauki,
jak i praktycznego ćwiczenia podstawowych koncepcji projektowania w interaktywnej formie,
łącząc aspekty edukacyjne z rozrywką.\\
\indent %Do realizacji projektu przyjęto architekturę MVVM (Model-View-ViewModel),
%która zapewnia modularność oraz przejrzystość kodu,
%umożliwiając łatwe utrzymanie i przyszłe rozszerzenia aplikacji.
Kluczowym elementem było opracowanie algorytmu weryfikującego
poprawność tworzonych schematów,
a także systemu poziomów,
opartego na cyklu: spełnienie wymagań poziomu – projektowanie układu – testowanie poprawności – ocenianie – \linebreak
- przejście
do kolejnego poziomu.
Poszczególne poziomy stanowią narastające wyzwania,
które pozwalają użytkownikowi rozwijać swoje umiejętności projektowe.\\
\indent Istotnym aspektem pracy było zaprojektowanie intuicyjnego edytora schema\-tów,
dostosowanego do potrzeb początkujących użytkowników.
W edytorze uwzględniono mechanizmy umożliwiające dynamiczne zarządzanie widocznością poszczególnych warstw i połączeń,
co ułatwia zrozumienie oraz kontrolę nad tworzonymi projektami.\\
\indent Efektem jest w pełni funkcjonalna gra edukacyjna,
która oferuje użytkownikom narzędzia do nauki projektowania układów scalonych
i umożliwia stopniowe podnoszenie poziomu trudności.
