\section*{Streszczenie}
\label{sec:streszczenie}
\thispagestyle{plain}
%\abstract
Celem niniejszej pracy jest opracowanie edukacyjnej gry komputerowej umożliwiającej wprowadzenie użytkownika
w zagadnienia projektowania schematów układów scalonych,
z wykorzystaniem silnika Unity i języka C\#.
Aplikacja stanowi narzędzie zarówno do nauki,
jak i praktycznego ćwiczenia podstawowych koncepcji projektowania w interaktywnej formie,
łącząc aspekty edukacyjne z rozrywką.\\
\indent %Do realizacji projektu przyjęto architekturę MVVM (Model-View-ViewModel),
%która zapewnia modularność oraz przejrzystość kodu,
%umożliwiając łatwe utrzymanie i przyszłe rozszerzenia aplikacji.
Kluczowym elementem było opracowanie algorytmu weryfikującego
poprawność tworzonych schematów oraz system poziomów o rosnącej trudności.
%opartego na cyklu: spełnienie wymagań poziomu – projektowanie układu – testowanie poprawności – ocenianie – \linebreak
%- przejście
%do kolejnego poziomu.
%Poszczególne poziomy stanowią narastające wyzwania,
%które pozwalają użytkownikowi rozwijać swoje umiejętności projektowe.\\
Innym istotnym aspektem pracy było zaprojektowanie intuicyjnego edytora schema\-tów,
dostosowanego do potrzeb początkujących użytkowników.
%W edytorze uwzględniono mechanizmy umożliwiające dynamiczne zarządzanie widocznością poszczególnych warstw i połączeń,
%co ułatwia zrozumienie oraz kontrolę nad tworzonymi projektami.\\
W edytorze zaimplementowano obsługę skrótów klawiszowych, oraz zaprojektowano intuicyjne w obsłudze narzę\-dzia rysowania i edycji,
zapewniające płynne i szybkie tworzenie schematów.\\
\indent Efektem jest funkcjonalna gra edukacyjna,
która oferuje użytkownikom narzę\-dzie do nauki projektowania układów scalonych
i umożliwia stopniowe doskonalenie umiejętności wraz z podnoszeniem poziomu trudności.\\
\\


\section*{Abstract}
\label{sec:abstract}

The aim of this thesis is to develop an educational computer game
that introduces users to the principles of designing integrated circuit schematics,
utilizing the Unity engine and the C\# programming language.
The application serves as a tool for both learning and practicing basic design concepts in an interactive format,
effectively combining educational and entertainment aspects.\\
\indent A key feature of the project is the development of an algorithm to verify the correctness of created schematics
and a level system with progressively increasing difficulty.
Another important aspect of the work is the design of an intuitive schematic editor tailored to the needs of novice users.
The editor integrates support for keyboard shortcuts and intuitive drawing and editing tools,
ensuring smooth and efficient schematic creation.\\
\indent The result is a functional educational game that provides users with a comprehensive tool
for learning integrated circuit design while
facilitating gradual skill improvement through progressively challenging levels.
