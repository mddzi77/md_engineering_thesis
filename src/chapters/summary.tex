\chapter{Podsumowanie}
\label{ch:podsumowanie}

Celem pracy było opracowanie programu edukacyjnego w~formie gry komputerowej,
umożliwiającego naukę projektowania układów scalonych.
%Część teoretyczna pracy opierałą się na analizie istniejących narzędzi do edycji schematów,
%porównaniu ich funkcjonalności, technologii rysowania oraz interfejsów użytkownika.
%Dla lepszego porównanie sprawdzone też inne popularne programy niebędące narzędziami do projektowania układów scalonych.
W~części teoretycznej przeanalizowano istniejące narzędzia do edycji schematów,
porównując ich funkcjonalności, technologie rysowania oraz interfejsy użytkownika.
Dodatkowo uwzględniono analizę innych popularnych programów graficznych.


%Stworzono takżę szereg założeń projektowych dotyczących zarówno aspektów technicznych, jak i~funkcjonalnych.
%Założenia techniczne obejmowały wybór technologii, wzorców projektowych oraz architektury aplikacji,
%które miały zapewnić efektywną implementację funkcji oraz łatwą rozbudowę projektu.
%Założenia funkcjonalne natomiast definiowały zakres funkcji, które miały być zaimplementowane w~aplikacji,
%takie jak obsługa warstw , mechanizmy rysowania i~edycji elementów układu,
%a~także system interakcji między użytkownikiem a~aplikacją.
%Założenia te umożliwiły określenie celów projektu oraz wyznaczenie kierunku jego rozwoju.
Sformułowano założenia projektowe obejmujące aspekty techniczne i~funkcjonalne.
Po stronie technicznej wybrano technologie,
wzorce projektowe oraz architekturę aplikacji,
zapewniającą efektywną implementację funkcji i~możliwość łatwej rozbudowy.
Funkcjonalnie zdefiniowano kluczowe elementy, takie jak obsługa warstw,
mechanizmy rysowania i~edycji elementów układu oraz system interakcji z~użytkownikiem.
Te założenia pozwoliły precyzyjnie określić cele projektu i~wyznaczyć jego kierunek rozwoju.



Przede wszystkim opracowano spójny i~intuicyjny edytor z~funkcjami rysowania,
modyfikacji i~usuwania elementów,
%pozwalających na tworzenie schematów układów scalonych,
umożliwiający łatwe tworzenie schematów układów scalonych przy szczególnym nacisku na prostotę obsługi.
%który pozwala na tworzenie i~edycję schematów układów scalonych.
%Zaimplementowano mechanizmy rysowania, modyfikacji oraz usuwania elementów, 
%w~których skupiono się na zapewnieniu prostoty obsługi i~czytelności interfejsu.
%Opracowano również system sprawdzania poprawności projektów,
%który pozwala na weryfikację zasadności połączeń między elementami układu.
%Na jego podstawie oparto część projektu dotyczącą gry edukacyjnej,
%która pozwala na zdobywanie praktycznych umiejętności w~projektowaniu układów scalonych.
Zaimplementowano także system weryfikacji poprawności projektów,
który umożliwia sprawdzanie połączeń między elementami.
Na tej podstawie opracowano mechanikę gry edukacyjnej,
pozwalającej użytkownikom zdobywać praktyczne umiejętności w~projektowaniu.
Rezultatem pracy jest program edukacyjny,
pozwalający na naukę projektowania układów scalonych w~sposób przystępny i~angażujący.


%Pomimo osiągnięcia głównych celów, projekt pozostawia przestrzeń na dalszy rozwój.
Pomimo realizacji głównych celów,
projekt oferuje szerokie możliwości dalszego rozwoju.
%Edytor można rozbudować o dodatkowe narzędzia edycji, jak na przykład obracanie elementów,
%czy kopiowanie.
Edytor można wzbogacić o dodatkowe funkcje,
takie jak obracanie czy kopiowanie elementów.
W~kontekście warstw warto zapewnić użytkownikowi większą kontrolę,
przykładowo poprzez opcję dodawania nowych warstw lub edycji ich właściwości.
Rozwój części edukacyjnej mógłby obejmować dodanie nowych poziomów,
umożliwiających użytkownikom zdobywanie bardziej zaawansowanych umiejętności.
%w~przypadku warstw można oddać większą kontrolę nad nimi użytkownikowi,
%przykładowo umożliwiając dodawanie nowych warstw lub zmianę ich właściwości.
%Sama część dotycząca gry edukacyjnej może być rozwijana o nowe poziomy,
%w~których użytkownik będzie mógł zdobywać coraz bardziej zaawansowane umiejętności.


Podsumowując, zrealizowany projekt stanowi solidną podstawę dla narzędzia edukacyjnego,
które może być rozwijane i~dostosowywane do potrzeb użytkowników, zarówno w~celach edukacyjnych,
jak~i~praktycznych.
Efekty tej pracy mogą posłużyć jako baza do dalszych badań
i~wdrożeń w~zakresie interaktywnych metod nauki projektowania układów scalonych.

%Proces ten obejmował stworzenie narzędzia,
%które łączy aspekty dydaktyczne z~interaktywnymi mechanizmami gry,
%co pozwala użytkownikom zdobywać praktyczne umiejętności w~zrozumiały i~angażujący sposób.
%w~projekcie wykorzystano silnik Unity,
%co umożliwiło efektywną implementację funkcji wspierających naukę projektowania
%i~ułatwiających przyswajanie wiedzy technicznej.\\
%\indent w~trakcie pracy zaimplementowano szereg funkcjonalności, takich jak obsługa warstw projektowych,
%mechanizmy rysowania i~modyfikacji elementów układu, a~także system interakcji między użytkownikiem a~aplikacją.
%Szczególnie ważnym aspektem była optymalizacja struktury kodu w~celu zapewnienia wydajności oraz czytelności,
%co zostało osiągnięte poprzez odpowiednie wykorzystanie wzorców projektowych
%oraz technologii takich jak UniTask do zarządzania operacjami asynchronicznymi.\\
%\indent Rezultatem pracy jest działający prototyp aplikacji,
%który został przetestowany pod kątem funkcjonalności oraz wydajności.
%Aplikacja ta umożliwia użytkownikom wykonywanie podstawowych operacji, takich jak tworzenie,
%edytowanie oraz zarządzanie elementami w~obrębie wielowarstwowego układu.
%Implementacja mechanizmu kolizji i~interakcji między warstwami pozwala na uniknięcie błędów projektowych
%i~wspiera proces nauki projektowania.\\
%\indent Pomimo osiągnięcia wielu założonych celów, projekt pozostawia przestrzeń na dalszy rozwój.
%w~przyszłych iteracjach możliwe jest dodanie bardziej zaawansowanych funkcji,
%takich jak symulacja działania układów w~czasie rzeczywistym,
%co pozwoli na jeszcze lepsze zrozumienie zasad ich funkcjonowania.
%Ponadto warto rozważyć integrację aplikacji z~innymi narzędziami edukacyjnymi
%oraz wsparcie dla bardziej złożonych układów scalonych, co mogłoby otworzyć nowe możliwości zastosowań dydaktycznych.\\
%\indent Podsumowując, zrealizowany projekt stanowi solidną podstawę dla narzędzia edukacyjnego,
%które może być rozwijane i~dostosowywane do potrzeb użytkowników, zarówno w~celach edukacyjnych,
%jak i~praktycznych.
%Efekty tej pracy mogą posłużyć jako baza do dalszych badań
%i~wdrożeń w~zakresie interaktywnych metod nauki projektowania układów scalonych.