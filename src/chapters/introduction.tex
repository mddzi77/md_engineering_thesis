\chapter{Wstęp i cel pracy}

Programy do tworzenia schematów są jednym z kluczowych elementów
procesu wielkoskalowej integracji układów scalonych (ang. VLSI — Very Large Scale Integration)~\cite{VLSI_insemi}.
Polega to na projektowaniu podukładów o określonych funkcjach,
które następnie są łączone w jeden, w pełni działający układ scalony.
Taki proces upraszcza projektowanie oraz produkcję układów scalonych, ponieważ prace można podzielić na mniejsze,
mniej skomplikowane części.
Dodatkowo wiele podukładów może być wykorzystywanych w różnych projektach,
co znacznie zwiększa efektywność pracy i obniża jej koszty.
W efekcie,
programy wspierające projektowanie schematów są nieodzownym narzędziem przy tworzeniu układów scalonych.
%W ten sposób upraszczany jest proces projektowania oraz produkcji układów scalonych,
%gdyż pracę można podzielić na mniejsze części,
%a dodatkowo cześć podukładów może być używana w wielu różnych projektach.
Przykładem takiego programu jest open-source'owy (o publicznym kodzie źródłowym),
MAGIC VLSI stworzony przez Johna Ousterhouta w 1980 roku~\cite{MAGIC},
napisany w języku C na platformę Linux\cite{MAGIC_wiki}.
Charakteryzuje go prosta szata graficzna oraz szeroki zakres działania,
lecz jego obsługa bywa często nieintuicyjna oraz nieprzyjazna dla początkujących użytkowników.
Kolejnym powszechnie dostępnym programem jest Microwind~\cite{Microwind},
niestety jego obsługa również nie należy do najprostszych, a część funkcji jest dostępna jedynie w płatnej wersji.
W wielu innych przypadkach programy te są dostępne jedynie w wersjach płatnych, kierowanych do dużych firm.
%co stało się inspiracją dla tego projektu.\\
To właśnie wysoki próg wejścia
i złożoność obsługi tego rodzaju programów stały się inspiracją dla stworzenia niniejszego projektu.\\
\indent Celem tego projektu było stworzenie
programu edukacyjnego w formie gry, który pomoże w nauce tworzenia schematów układów scalonych.
Jedne z głównych założeń to prostota obsługi, intuicyjność oraz ergonomia,
na które mają wpływ przede wszystkim dobrze zaprojektowany interfejs graficzny użytkownika
oraz wbudowane narzędzia wspierające edycję schematu.
%Na tych założeniach opiera się cała część projektu związana z interfejsem użytkownika.\\
Funkcje te stanowią podstawę części projektu związaną z interfejsem użytkownika
i odpowiadają za poprawę użyteczności oraz dostępność aplikacji.\\
\indent Kolejnym istotnym elementem tej pracy było przygotowanie angażującej i skalowalnej pętli rozgrywki,
która pozwoli na stopniowe wprowadzanie użytkownika w proces projektowania schematów układów scalonych,
wraz z implementacją systemów sprawdzania poprawności wykonanych zadań, wskazywania błędów i ewentualnych podpowiedzi.
%Dzięki realizacji założeń program pomoże w nauce podstaw bez odrzucania użytkownika
%przez zbyt skomplikowaną obsługę.\\
Dzięki realizacji założeń, projekt ten pozwoli użytkownikowi na zdobycie podstawowej wiedzy
i umiejętności z zakresu projektowania układów scalonych,
bez konieczności przechodzenia przez trudniejsze w obsłudze, tradycyjne oprogramowanie.\\
\indent Zważając na wszystkie założenia oraz charakter edukacyjno-growy projektu,
do realizacji wybrano silnik Unity Engine,
bazujący na języku C\#,
powszechnie wykorzystywany w tworzenia gier komputerowych.
%program został stworzony w silniku Unity Engine,
%powszechnie używanym do tworzenia gier komputerowych.
%natomiast możliwość łatwego tworzenia interaktywnej aplikacji graficznej sprawdzi się w tego typu zastosowaniach.\\
Wybór Unity podyktowany był prostotą tworzenia aplikacji graficznych oraz łatwego tworzenia interaktywnych elementów,
co można wykorzystać nie tylko w grach, ale również w różnego typu narzędziach.
Dzięki Unity,
program łączy funkcje gry z elementami edukacyjnymi,
tworząc przystępną i atrakcyjną platformę do nauki podstaw projektowania układów scalonych.
%Został on wybrany ze względu na prostotę tworzenie aplikacji graficznych oraz łatwego tworzenia interaktywnych elementów,
%co można wykorzystać nie tylko w grach, ale również w różnego typu narzędziach.
%Aby ujednolicić schematy oraz zadania do wykonania dla użytkowników,
%projektowanie będzie odbywać się w tylko jednej technologii\ \textendash \ CMOS AMIS ami-C5.
%Powodem wyboru tej technologii jest wcześniejsze doświadczenie w projektowaniu schematów układów scalonych
%w tej technologii.
