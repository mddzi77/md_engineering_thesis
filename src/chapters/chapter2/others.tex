\section{Inne programy}
\label{sec:inne_programy}

Podczas opracowywania oprogramowania,
które spełniać będzie obecne standardy ergonomii i~intuicyjności,
warto zwrócić uwagę na inne popularne programy z~odrębnych kategorii.
W~szczególności będącym na darmowej licencji,
przez~co~są dostępne dla większej liczby użytkowników
i~idealne do nauki danej dziedziny, której dotyczą.

\subsection{GIMP}
\label{subsec:gimp}

Podobnie jak Electric, GIMP jest oficjalnym pakietem GNU i~również posiada otwarty kod źródłowy
oraz~stosuje założenia wolnego oprogramowania~\cite{gimp_site}.
Jest dostępny na wiele platform, w~tym na systemy Windows, Linux oraz~macOS.
GIMP to program do edycji grafiki rastrowej.
Charakteryzuje się przejrzystym interfejsem graficznym oraz~dużą ilością narzędzi,
gdzie każde z~nich ma swój własny skrót klawiszowy, co~znacznie ułatwia pracę w~programie.
\indent Interfejs graficzny jest podzielony na trzy główne panele, nie licząc obszaru roboczego:

\begin{citemize}
    \item Górny panel menu programu -- dostępne opcje dla programu oraz~dodatkowe opcje do modyfikacji edytowanego obrazu,
    \item Prawy panel -- zawiera przede wszystkim panel do zarządzania warstwami obrazu,
    ale również szereg innych opcji takich jak krzywe, desenie czy wybór pędzli
    \item Lewy panel narzędzi -- zawiera narzędzia do rysowania oraz modyfikacji obrazu,
    a~także opcje tych narzędzi.
\end{citemize}

Widok okna programu GIMP przedstawiono na rys.~\ref{fig:gimp_okno}.
Dzięki zastosowaniu takiego interfejsu program jest przejrzysty dzięki zgrupowaniu konkretnych typów funkcjonalności,
co~może mieć także zastosowanie w~opracowywanym programie. \\
\indent Dzięki darmowemu dostępowi i~szerokiej funkcjonalności,
GIMP został także wykorzystany do opracowania grafik do niniejszej pracy.

\begin{figure}[h]
    \centering
    \includegraphics[width=.9\textwidth]{chapters/chapter2/img/gimp}
    \caption[Widok głównego okna programu GIMP.]{Widok głównego okna programu GIMP, źródło:~\cite{gimp_site}.}
    \label{fig:gimp_okno}
\end{figure}

\subsection{Blender}
\label{subsec:blender}

Program Blender jest kolejnym przykładem oprogramowania z~otwartym kodem źródłowym na licencji GNU~\cite{blender_site}.
Wykorzystywany jest do tworzenia grafiki trójwymiarowej, animacji oraz~efektów specjalnych.
Ze względu na rozbudowane możliwości oraz~ilość narzędzi i~funkcji,
zasadnicze znaczenie ma ergonomia i~przejrzystość interfejsu, pokazanego na rys.~\ref{fig:blender_okno}.
Poza podobnymi jak w~reszcie programów panelami górnym i~prawym,
Blender wyróżnia się dolnym panelem z~osią czasu do animacji, a~także małym paskiem narzędzi w~lewym górnym rogu okna,
co~jest efektywnym sposobem pozwalającym na zaoszczędzenie miejsca na obszar roboczy,
gdy nie ma dużej liczby narzędzi do wykorzystania w~danym momencie.
Podobne rozwiązanie może znaleźć zastosowanie w~edytorze schematów.

\begin{figure}[h]
    \centering
    \includegraphics[width=.9\textwidth]{chapters/chapter2/img/blender}
    \caption[Widok głównego okna programu Blender.]{Widok głównego okna programu Blender, źródło:~\cite{blender_site}.}
    \label{fig:blender_okno}
\end{figure}
