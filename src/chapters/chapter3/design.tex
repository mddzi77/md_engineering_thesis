\section{Założenia funkcjonalne}
\label{sec:zalozenia_funkcjonalne}

Najistotniejsze podczas projektowania programu są założenia funkcjonalne,
które tak naprawdę definiują późniejsze założenia techniczne oraz samo działanie aplikacji.
Założenia te definiują obszary takie jak interfejs użytkownika, mechanizmy oraz wprowadzanie danych wejściowych.\\
\indent Głównym celem aplikacji jest umożliwienie użytkownikowi projektowanie prostych schematów układów scalonych.
Wymaga to zaimplementowania edytora graficznego będącego w stanie przedstawić przystępnie topografię układu scalonego.
Najbardziej intuicyjne jest wykorzystanie podejścia opierającego się na rysowaniu komórek,
które składałyby się w elementy tworzące schemat, podobnie jak to ma miejsce w programach Magic VLSI czy Microwind.
Aby wprowadzić podstawowe założenia projektowania układu scalonego, oparto je o technologie Amis C5,
opracowane przez firmę Mosis.
Technologia ta definiuje rzeczywiste wymiary każdej kwadratowej komórki poprzez parametr $\lambda$
zawarty w regułach projektowania.
Sam wybór jakiejkolwiek technologii projektowania podyktowany
jest próbą nauczenia skalowania wymiarów komórek względem wymiarów rzeczywistych.
Jedne z takich reguł projektowania to \textit{SCMOS\_SUBM} o $\lambda=0,3~\um$~\cite{amis_c5, amis_params}.
Definiują one również dostępne materiały, z których może się składać układ, parametry elementów oraz ograniczenia.
Podstawowe materiały przedstawiono w tab.~\ref{tab:amis_materials}.
Wybór tych materiałów powinien odbywać się poprzez paletę warstw,
dzięki czemu użytkownik może\linebreak w sposób prosty wybrać odpowiedni z nich,
jednocześnie mając przegląd wszystkich dostępnych warstw. 
%Zasady projektowania wskazują także najmniejsze dopuszczalne tranzystory: 3,0\um \times 0,6\um (W/L)~\cite{amis_params}.,
%czyli po podzieleniu przez $\lambda$ otrzymujemy 10 \times 2.

\newpage
\begin{table}[h]
    \centering
    \caption[Dostępne materiały w technologii Amis C5.]
    {Dostępne podstawowe materiały w technologii Amis C5, źródło:~\cite{amis_params}.}
    \label{tab:amis_materials}
    \begin{tabular}{|l|l|}
        \hline
        Nazwa materiału & Opis \\
        \hline
        \hline
        \texttt{substrate} & podłoże \\
        \hline
        \texttt{N+active} & inaczej \texttt{ndiffusion}, krzem typu n\\
        \hline
        \texttt{P+active} & inaczej \texttt{pdiffusion}, krzem typu p\\
        \hline
        \texttt{poly} & polikrystaliczny krzem \\
        \hline
        \texttt{poly2} & druga warstwa polikrystalicznego krzemu \\
        \hline
        \texttt{metal1} & pierwsza warstwa metalu \\
        \hline
        \texttt{metal2} & druga warstwa metalu \\
        \hline
    \end{tabular}
\end{table}

Należy zaznaczyć również że poza samymi warstwami istotne jest także wprowadzenie połączeń między nimi.
Kolejnym istotnym aspektem jest dobór narzędzi edycji schematu.
Powinny one być dostępne w sposób intuicyjny, być łatwe w obsłudze oraz umożliwiać większość operacji na schemacie.
Narzędzia te można rozdzielić na dwa rodzaje: narzędzia rysujące oraz narzędzia edytujące. \\
\indent Poza samym edytorem istotne jest także wprowadzenie mechaniki gry.\linebreak
Ma ona na celu nauczenie użytkownika podstaw projektowania układów scalonych,
stąd też powinna być maksymalnie prosta tak, aby można było skupić się na nauce\linebreak i na samym,
dość skomplikowanym, procesie projektowania

\subsection{Narzędzia rysujące}
\label{subsec:narzedzia_rysujace}

Rysowanie jest podstawą tworzenia schematu, przez co powinny one być najprostsze i najbardziej intuicyjne.
Każde z narzędzi powinno mieć przypisane skróty klawiszowe,
które przedstawiono wraz z resztą w tab.~\ref{tab:key_shortcuts}. % TODO: dodać coś o warstwach i ich wyborze0
Z podstawowych narzędzi, jakie mogą być przydatne w edytorze schematów, można wymienić:

\begin{citemize}
    \item Pędzel -- podstawowe narzędzie rysujące, które pozwala na rysowanie pojedynczych komórek
    i które ma największą swobodę i precyzję rysowania,
    \item Prostokąt -- narzędzie rysujące prostokątne obszary, które pozwala na szybkie,
    ale i precyzyjne rysowanie dzięki dwóm trybom,
    \item Gumka -- działanie podobne do pędzla, ale zamiast rysować, usuwa komórki.
\end{citemize}

\indent Proces rysowania jest podobny do tego w programach graficznych,
dzięki oparciu o rysowanie komórek, co pomaga w łatwym zapoznaniu się z edytorem.\linebreak
Przed rysowaniem wybiera się warstwę (podobnie jak kolor w programach graficznych) oraz narzędzie rysujące.

\subsection{Narzędzia edytujące}
\label{subsec:narzedzia_edytujace}

Narzędzia edytujące pozwalają na modyfikowanie już narysowanych komórek.
Podstawą jest zaznaczanie komórek do edytowania,
stąd potrzeba narzędzia zaznaczającego.
Opierać się ono powinno na takiej samej zasadzie działania jak narzędzie rysowania prostokątnego obszaru,
dzięki czemu użytkownik nie musi zapamiętywać kilku różnych mechanizmów.
Dodatkowo powinny istnieć skróty klawiszowe pozwalające na samo modyfikowanie obszaru poprzez dodawanie
oraz odejmowanie zaznaczenia, przedstawiono je w tab.~\ref{tab:key_shortcuts}.
Zaznaczony obszar można następnie modyfikować, poprzez narzędzie przesuwania.
Operacje te, również będą miały swoje przypisania klawiszy, co pozwoli na szybsze operowanie na schemacie. \\
\indent Istotnymi funkcjami jest też oznaczanie istotnych miejsc na schemacie, takich jak kontakty między warstwami,
czy wejście zasilania i masy.
Mimo tego, że są one elementami samego schematu i mogą być narysowane,
to powinny być one wyróżnione poprzez skategoryzowanie jako odrębne narzędzia.
%\indent Istotną funkcją jest także możliwość cofania ostatnich operacji,
%natomiast ze względu na prostotę edytora ilość cofnięć, powinna być ograniczona,
%co pozwoli także na uproszczenie programu od strony technicznej. % TODO: dodać ilość kroków cofania

\begin{table}[h]
    \centering
    \caption[Skróty klawiszowe używane w edytorze schematów.]
    {Skróty klawiszowe używane w edytorze schematów, źródło: opracowanie własne.}
    \label{tab:key_shortcuts}
    \begin{tabular}{|l||l|p{0.66\textwidth}|}
        \hline
        Skrót & \multicolumn{2}{|l|}{Opis} \\
        \hline
        \hline
        \texttt{B} & \multicolumn{2}{|l|}{Pędzel} \\
        \hline
        \texttt{R} & \multicolumn{2}{|l|}{Rysowanie prostokątnego obszaru poprzez przeciąganie} \\
        \cline{2-3}
        & \texttt{Shift} & Przytrzymanie pozwala na rysowanie poprzez wybranie punktu początkowego i końcowego obszaru \\
        \hline
        \texttt{E} & \multicolumn{2}{|l|}{Gumka} \\
        \hline
        \texttt{W} & \multicolumn{2}{|l|}{Usuwanie prostokątnego obszaru} \\
        \cline{2-3}
        & \texttt{Shift} & Przytrzymanie pozwala na rysowanie poprzez wybranie punktu początkowego i końcowego obszaru \\
        \hline
        \texttt{S} & \multicolumn{2}{|l|}{Zaznaczanie obszaru poprzez przeciąganie} \\
        \cline{2-3}
        & \texttt{Shift} & Przytrzymanie pozwala na rysowanie poprzez wybranie punktu początkowego i końcowego obszaru \\
        \cline{2-3}
        & \texttt{Ctrl} & Przytrzymanie pozwala na dodawanie zaznaczenia \\
        \cline{2-3}
        & \texttt{Alt} & Przytrzymanie pozwala na odejmowanie zaznaczenia \\
        \hline
        \texttt{M} & \multicolumn{2}{|l|}{Przesuwanie zaznaczonego obszaru} \\
        \hline
        \texttt{C} & \multicolumn{2}{|l|}{Zaznaczanie kontaktów} \\
        \cline{2-3}
        & \texttt{Shift} & Przytrzymanie pozwala na rysowanie poprzez wybranie punktu początkowego i końcowego obszaru \\
        \hline
        \texttt{D} & \multicolumn{2}{|l|}{Zaznaczanie zasilania} \\
        \hline
        \texttt{G} & \multicolumn{2}{|l|}{Zaznaczanie masy} \\
        \hline
%        \hline
%        \texttt{Ctrl + Z} & \multicolumn{2}{|l|}{Cofanie ostatniej operacji} \\
%        \hline
%        \texttt{Ctrl + Y} & \multicolumn{2}{|l|}{Ponawianie cofniętej operacji} \\
%        \hline
%        \texttt{Delete} & \multicolumn{2}{|l|}{Usuwanie zaznaczonego elementu} \\
%        \hline
%        \texttt{Ctrl + X} & \multicolumn{2}{|l|}{Wycinanie zaznaczonego elementu} \\
%        \hline
%        \texttt{Ctrl + C} & \multicolumn{2}{|l|}{Kopiowanie zaznaczonego elementu} \\
%        \hline
%        \texttt{Ctrl + V} & \multicolumn{2}{|l|}{Wklejanie skopiowanego/wyciętego elementu} \\
%        \hline
    \end{tabular}
\end{table}\

%\begin{table}[h]
%    \centering
%    \caption[Skróty klawiszowe używane w edytorze schematów.]
%    {Skróty klawiszowe używane w edytorze schematów, źródło: opracowanie własne.}
%    \label{tab:key_shortcuts}
%    \begin{tabular}{l l p{0.7\textwidth}}
%        Skrót & \multicolumn{2}{l}{Opis} \\
%        \hline
%        \texttt{B} & \multicolumn{2}{l}{Pędzel} \\
%        \texttt{R} & \multicolumn{2}{l}{Rysowanie prostokątnego obszaru poprzez przeciąganie} \\
%        & \texttt{Shift} & Przytrzymanie pozwala na rysowanie poprzez wybranie punktu początkowego i końcowego obszaru \\
%        \texttt{S} & \multicolumn{2}{l}{Zaznaczanie obszaru poprzez przeciąganie} \\
%        & \texttt{Shift} & Przytrzymanie pozwala na rysowanie poprzez wybranie punktu początkowego i końcowego obszaru \\
%        & \texttt{Ctrl} & Dodawanie zaznaczenia \\
%        & \texttt{Alt} & Odejmowanie zaznaczenia \\
%    \end{tabular}
%\end{table}

\subsection{Mechanika gry}
\label{subsec:mechanika_gry}

Mechanika gry ma na celu nauczenie użytkownika podstaw projektowania układów scalonych,
jednocześnie starając się nie skomplikować samego procesu projektowania.
Tak samo, jak w grach typu puzzle, gra będzie składać się z poziomów,
które będą stopniowo wprowadzać użytkownika w proces projektowania.
Każdy z poziomów będzie posiadał wymagania w formie listy komponentów,
gdzie powinny być podłączone oraz jakie mają mieć parametry.
%w formie schematu elektrycznego oraz opisowej instrukcji.
Gdy użytkownik uzna, że spełnił wymagania, będzie mógł sprawdzić swoje rozwiązanie,
które zostanie zweryfikowane przez system sprawdzający.
W przypadku błędów %punktacja za poziom będzie obniżona,
użytkownik będzie mógł poprawić swoje rozwiązanie. %lub rozpocząć poziom od nowa z pełną pulą punktów.
%Dodatkowo w trakcie projektowania będzie można sprawdzić, czy dotychczasowy schemat jest poprawny,
%kosztem finalnej punktacji za poziom.
%Aby uniemożliwić nadużywanie tej mechaniki, wprowadzony zostanie limit czasu na jej użycie.
Gdy zatwierdzony schemat spełni wymagania, gracz może przejść do kolejnego poziomu.
Przepływ rozgrywki dokładniej przedstawiono w tab.~\ref{tab:game_flow}.

\begin{table}[h]
    \centering
    \caption[Przepływ rozgrywki.]
    {Przepływ rozgrywki, źródło: opracowanie własne.}
    \label{tab:game_flow}
    \begin{tabular}{|l p{0.22\textwidth}|p{0.6\textwidth}|}
        \hline
        \multicolumn{2}{|l|}{Krok} & Opis \\
        \hline
        \hline
        1 & Rozpoczęcie poziomu &
        Użytkownik rozpoczyna poziom, przedstawia się mu wymagania w formie listy wymaganych komponentów. \\
        \hline
        2 & Projektowanie schematu &
        Użytkownik projektuje schemat\\ %, w ograniczonym czasie może sprawdzać jego poprawność kosztem punktów. \\
        \hline
        3 & Weryfikacja schematu &
        Użytkownik zatwierdza schemat, który następnie zostaje sprawdzony przez system. \\
        \hline
        4 & Wynik &
        Gracz dostaje informacje czy wymagania zostały spełnione, czy nie. \\
        \hline
        5a & Wynik pozytywny &
        Wymagania zostały spełnione, gracz może przejść do kolejnego poziomu. \\
        \hline
        5b & Wynik negatywny &
        Wymagania nie zostały spełnione, gracz może poprawić schemat. \\
        \hline
    \end{tabular}
\end{table}

%\begin{table}[h]
%    \centering
%    \caption[Przepływ rozgrywki.]
%    {Przepływ rozgrywki, źródło: opracowanie własne.}
%    \label{tab:game_flow}
%    \begin{tabular}{l p{0.22\textwidth} p{0.6\textwidth}}
%        \multicolumn{2}{l}{Krok} & Opis \\
%        \hline
%        1 & Rozpoczęcie poziomu &
%        Użytkownik rozpoczyna poziom, przedstawia się mu wymagania w formie schematu elektrycznego oraz opisu. \\
%        2 & Projektowanie schematu & 
%        Użytkownik projektuje schemat, w ograniczonym czasie może sprawdzać jego poprawność kosztem punktów. \\
%        3 & Weryfikacja schematu &
%        Użytkownik zatwierdza schemat, który następnie zostaje sprawdzony przez system. \\
%        4 & Wynik &
%        Gracz dostaje informacje czy wymagania zostały spełnione, czy nie. \\
%        5a & Wynik pozytywny &
%        Wymagania zostały spełnione, gracz może przejść do kolejnego poziomu. \\
%        5b & Wynik negatywny &
%        Wymagania nie zostały spełnione, gracz może poprawić schemat lub rozpocząć poziom od nowa. \\
%    \end{tabular}
%\end{table}

%\subsection{Podsumowanie założeń funkcjonalnych}
%\label{subsec:podsumowanie_zalozen_funkcjonalnych}
%
%Założenia funkcjonalne są kluczowe dla projektu, gdyż definiują one obszary, na które należy zwrócić szczególną uwagę.

